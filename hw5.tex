% <--- percent sign starts a comment line in LaTeX

%----------------------------------------------------------
% This is a sample assignment .tex file. Put your name,
% assignment number and the due date below, as shown.
% Before you typeset your own assignment try to preview 
% and print this one as follows:
%	0. Make sure your LaTeX is installed
%   1. Save this in a file, say HwSample.tex
%	2. Save macros.tex (and other source files, if needed)
%	3. Check the references to other source files in this file, make sure
%		that they are in correct directories. Typically, I will have figures
%		in the same directory, while macros.tex two levels up. But you can
%		change it as you wish.
%   4. Run LaTeX on HwSample.tex
%	5. LaTeX will produce several files, including one file HwSample.pdf
%	6. Use your pdf viewer to view/print HwSample.pdf
% ------------------------------------------------------------

\documentclass[11pt]{article}

\usepackage{fullpage,graphicx,latexsym,picinpar,amsbsy,amsmath,amsfonts}

           

%%%%%%%%%%%%%%%%%%%%%%%%%%%%%%%%%%%%%%%%%%%%%%%%%%%%%%%%%%%%%%%%%%%%%%%%%%%%%%%%%%%
%%%%%%%%%%%  LETTERS 
%%%%%%%%%%%%%%%%%%%%%%%%%%%%%%%%%%%%%%%%%%%%%%%%%%%%%%%%%%%%%%%%%%%%%%%%%%%%%%%%%%%

\newcommand{\barx}{{\bar x}}
\newcommand{\bary}{{\bar y}}
\newcommand{\barz}{{\bar z}}
\newcommand{\bart}{{\bar t}}

\newcommand{\bfP}{{\bf{P}}}

%%%%%%%%%%%%%%%%%%%%%%%%%%%%%%%%%%%%%%%%%%%%%%%%%%%%%%%%%%%%%%%%%%%%%%%%%%%%%%%%%%%
%%%%%%%%%%%%%%%%%%%%%%%%%%%%%%%%%%%%%%%%%%%%%%%%%%%%%%%%%%%%%%%%%%%%%%%%%%%%%%%%%%%
                                                                                
\newcommand{\parend}[1]{{\left( #1  \right) }}
\newcommand{\spparend}[1]{{\left(\, #1  \,\right) }}
\newcommand{\angled}[1]{{\left\langle #1  \right\rangle }}
\newcommand{\brackd}[1]{{\left[ #1  \right] }}
\newcommand{\spbrackd}[1]{{\left[\, #1  \,\right] }}
\newcommand{\braced}[1]{{\left\{ #1  \right\} }}
\newcommand{\leftbraced}[1]{{\left\{ #1  \right. }}
\newcommand{\floor}[1]{{\left\lfloor #1\right\rfloor}}
\newcommand{\ceiling}[1]{{\left\lceil #1\right\rceil}}
\newcommand{\barred}[1]{{\left|#1\right|}}
\newcommand{\doublebarred}[1]{{\left|\left|#1\right|\right|}}
\newcommand{\spaced}[1]{{\, #1\, }}
\newcommand{\suchthat}{{\spaced{|}}}
\newcommand{\numof}{{\sharp}}
\newcommand{\assign}{{\,\leftarrow\,}}
\newcommand{\myaccept}{{\mbox{\tiny accept}}}
\newcommand{\myreject}{{\mbox{\tiny reject}}}
\newcommand{\blanksymbol}{{\sqcup}}
                                                                                                                         
\newcommand{\veps}{{\varepsilon}}
\newcommand{\Sigmastar}{{\Sigma^\ast}}
                           
\newcommand{\half}{\mbox{$\frac{1}{2}$}}    
\newcommand{\threehalfs}{\mbox{$\frac{3}{2}$}}   
\newcommand{\domino}[2]{\left[\frac{#1}{#2}\right]}  

%%%%%%%%%%%% complexity classes

\newcommand{\PP}{\mathbb{P}}
\newcommand{\NP}{\mathbb{NP}}
\newcommand{\PSPACE}{\mathbb{PSPACE}}
\newcommand{\coNP}{\textrm{co}\mathbb{NP}}
\newcommand{\DLOG}{\mathbb{L}}
\newcommand{\NLOG}{\mathbb{NL}}
\newcommand{\NL}{\mathbb{NL}}

%%%%%%%%%%% decision problems

\newcommand{\PCP}{\sc{PCP}}
\newcommand{\Path}{\sc{Path}}
\newcommand{\GenGeo}{\sc{Generalized Geography}}

\newcommand{\malytm}{{\mbox{\tiny TM}}}
\newcommand{\malycfg}{{\mbox{\tiny CFG}}}
\newcommand{\Atm}{\mbox{\rm A}_\malytm}
\newcommand{\complAtm}{{\overline{\mbox{\rm A}}}_\malytm}
\newcommand{\AllCFG}{{\mbox{\sc All}}_\malycfg}
\newcommand{\complAllCFG}{{\overline{\mbox{\sc All}}}_\malycfg}
\newcommand{\complL}{{\bar L}}
\newcommand{\TQBF}{\mbox{\sc TQBF}}
\newcommand{\SAT}{\mbox{\sc SAT}}

%%%%%%%%%%%%%%%%%%%%%%%%%%%%%%%%%%%%%%%%%%%%%%%%%%%%%%%%%%%%%%%%%%%%%%%%%%%%%%%%%%%
%%%%%%%%%%%%%%% for homeworks
%%%%%%%%%%%%%%%%%%%%%%%%%%%%%%%%%%%%%%%%%%%%%%%%%%%%%%%%%%%%%%%%%%%%%%%%%%%%%%%%%%%

\newcommand{\student}[2]{%
{\noindent\Large{ \emph{#1} SID {#2} } \hfill} \vskip 0.1in}

\newcommand{\assignment}[1]{\medskip\centerline{\large\bf CS 111 ASSIGNMENT {#1}}}

\newcommand{\duedate}[1]{{\centerline{due {#1}\medskip}}}     

\newcounter{problemnumber}                                                                                 

\newenvironment{problem}{{\vskip 0.1in \noindent
              \bf Problem~\addtocounter{problemnumber}{1}\arabic{problemnumber}:}}{}

\newcounter{solutionnumber}

\newenvironment{solution}{{\vskip 0.1in \noindent
             \bf Solution~\addtocounter{solutionnumber}{1}\arabic{solutionnumber}:}}
				{\ \newline\smallskip\lineacross\smallskip}

\newcommand{\lineacross}{\noindent\mbox{}\hrulefill\mbox{}}

\newcommand{\decproblem}[3]{%
\medskip
\noindent
\begin{list}{\hfill}{\setlength{\labelsep}{0in}
                       \setlength{\topsep}{0in}
                       \setlength{\partopsep}{0in}
                       \setlength{\leftmargin}{0in}
                       \setlength{\listparindent}{0in}
                       \setlength{\labelwidth}{0.5in}
                       \setlength{\itemindent}{0in}
                       \setlength{\itemsep}{0in}
                     }
\item{{{\sc{#1}}:}}
                \begin{list}{\hfill}{\setlength{\labelsep}{0.1in}
                       \setlength{\topsep}{0in}
                       \setlength{\partopsep}{0in}
                       \setlength{\leftmargin}{0.5in}
                       \setlength{\labelwidth}{0.5in}
                       \setlength{\listparindent}{0in}
                       \setlength{\itemindent}{0in}
                       \setlength{\itemsep}{0in}
                       }
                \item{{\em Instance:\ }}{#2}
                \item{{\em Query:\ }}{#3}
                \end{list}
\end{list}
\medskip
}

%%%%%%%%%%%%%%%%%%%%%%%%%%%%%%%%%%%%%%%%%%%%%%%%%%%%%%%%%%%%%%%%%%%%%%%%%%%%%%%%%%%
%%%%%%%%%%%%% for quizzes
%%%%%%%%%%%%%%%%%%%%%%%%%%%%%%%%%%%%%%%%%%%%%%%%%%%%%%%%%%%%%%%%%%%%%%%%%%%%%%%%%%%

\newcommand{\quizheader}{ {\large NAME: \hskip 3in SID:\hfill}
                                \newline\lineacross \medskip }


%%%%%%%%%%%%%%%%%%%%%%%%%%%%%%%%%%%%%%%%%%%%%%%%%%%%%%%%%%%%%%%%%%%%%%%%%%%%%%%%%%%
%%%%%%%%%%%%% for final
%%%%%%%%%%%%%%%%%%%%%%%%%%%%%%%%%%%%%%%%%%%%%%%%%%%%%%%%%%%%%%%%%%%%%%%%%%%%%%%%%%%

\newcommand{\namespace}{\noindent{\Large NAME: \hfill SID:\hskip 1.5in\ }\\\medskip\noindent\mbox{}\hrulefill\mbox{}}



\begin{document}
	
% v -- YOUR NAME and SID in the braces
\student{Krzysztof Brz\c eczyszczykiewicz}{834322002}  
% v -- YOUR NAME and SID in the braces
\student{Venkataratnam Narasimha Rattaiah}{854377890} 
% v -- ASSIGNMENT NUMBER in the braces
\assignment{127} 
% v-- DUE DATE in the braces
\duedate{Friday, February 13}  

\medskip

%%%%%%%%%%%%%%%%%%%%%%%%%%%%%%%%%%%%%%%%%%%%%%%%%%%%%%%%%%%%%%%%%%%%%%%%%%

\lineacross

%%%%%%%%%%%%%%%%%%%%%%%%%%%%

\begin{problem}
Consider a sequence defined recursively as
$a_0 = 4$, $a_1 = 9$, and $a_n = a_{n-1}+3a_{n-2}$ for
$n\ge 2$. Prove that $a_n = O(2.5^n)$.
\end{problem}

%---------------------------

\begin{solution}
We first prove by induction that $a_n \le 4(2.5)^n$ is true.

\noindent
\emph{Base step:} $a_0 = 4,\;4(2.5)^0=4$, so we have $a_0 \le 4(2.5)^0$.

\noindent
\emph{Inductive step:} The inductive hypothesis is that
$a_n \le 4(2.5)^n$ is true for $n=0,1,...,k$.
To complete the inductive step, we need to show that $a_{k+1}\le 4(2.5)^{k+1}$.

Since $a_k \le 4(2.5)^k$ and $a_{k-1} \le 4(2.5)^{k-1}$ holds, we have
%
\begin{eqnarray*}
a_{k+1}  &=& a_{k} + 3a_{k-1}
                \\
         &\le& 4(2.5)^k + 3\cdot4(2.5)^{k-1}
                \\
        &\le& 4(2.5)^{k-1}(2.5+3)
                \\
        &\le& 4(2.5)^{k-1}(2.5)^2
                \\
        &\le& 4(2.5)^{k+1}.
\end{eqnarray*}
%
That is, we have shown that if the hypothesis is true,
then $a_{k+1}\le 4(2.5)^{k+1}$ is also true.
This complete the inductive step.

We thus have $a_n \le C \vert{(2.5)^n}\vert $ for $n \ge 0$ and
$C = 4$. Therefore $a_n = O(2.5^n)$.
\end{solution}

%%%%%%%%%%%%%%%%%%%%%%%%%%%%

\begin{problem}
Suppose we have three sets, $A_1$, $A_2$, $A_3$
with the following properties:
%
\begin{description}

\item{(a)} $|A_2| = 2|A_1|$, $|A_3| = 4|A_1|$,
               
\item{(b)} $|A_1\cap A_2| = 2 $,
        $|A_1\cap A_3| = 2$,
        $|A_2\cap A_3| = 5$,

\item{(c)}
$|A_1\cap A_2\cap A_3| = 1$

\item{(d)}
$|A_1\cup A_2\cup A_3| = 27$.

\end{description}
%
Use the inclusion-exclusion formula to
determine the number of elements in $A_1\cup A_1\cup A_3$.
Show your work.
\end{problem}

%-----------------------------

\begin{solution}
We have
\begin{eqnarray*}
        |A_1\cup A_2\cup A_3| &=& |A_1| +|A_2| + |A_3| -|A_1\cap A_2| -|A_1\cap A_3| -|A_2\cap A_3| + |A_1\cap A_2\cap A_3| \\
        27 &=& |A_1| + |A_2| + |A_3| - 2 - 2 - 5 + 1 \\
        |A_1| + |A_2| + |A_3| &=& 35 \\
	|A_2| &=& 2 |A_1| \\
	|A_3| &=& 4 |A_1| 
\end{eqnarray*}
By solving the above equations, we get the cardinalities of $A_1$, $A_2$ and $A_3$:
\begin{eqnarray*}
  |A_1| &=& 5\\
  |A_2| &=& 10\\
  |A_3| &=& 20
\end{eqnarray*}
\end{solution}

%%%%%%%%%%%%%%%%%%%%%%%%%%%

\begin{problem}
We are given an array $A[0,...,n-1]$ that contains distinct numbers sorted 
in increasing order, that is $A[i] < A[i+1]$ for all $i=0,...,n-2$.
Consider the algorithm below, described in pseudo-code

% the verbatim environment is one option for formatting pseudo-code of algorithms,
% although there are better ways to do that 
\begin{center}
\begin{minipage}{3in}
\begin{verbatim}
1	i <- 0;
2	j <- 0;
3	count <- 0;
4	while i < n do
5	     while 2*A[j] < A[i] 
6	          do j <- j+1;
7	     if 2*A[j] = A[i] 
8	          then count <- count+1;
9	     i <- i+1;
10	print(count);
\end{verbatim}
\end{minipage}
\end{center}

Explain what value is computed by this algorithm and
give an asymptotic running time for this algorithm.
Justify your answer. 
\end{problem}


%----------------------------

\begin{solution}
(b) If all the numbers are non-negative, then
this program computes the number of those $i$ for which
$A[i]/2$ is also in the array. (Equivalently, it's the number of
$j$'s for which $2A[j]$ is in the array.)
The correctness follows from the fact that 
the entries in $A[]$ are strictly increasing. A more
detailed explanation follows.

The formal proof can be carried out using mathematical induction. 
We claim that for each $i$, after the internal while loop stops, $j$
is the smallest index for which $A[i] \le 2A[j]$.

Indeed, for $i=0$ we will end up with $j=0$, which
satisfies the claim. For the inductive step,
suppose that this is true for some $i$.
Then the smallest $j'$ for which $A[i+1] \le 2A[j']$
is at least as large as $j$ and cannot be greater
than $i+1$. This means that the internal while loop will
correctly find this $j'$ because it tries all these
values in increasing order and stops when it finds the
first one that satisfies this condition. Thus the claim above
holds.

The program starts with $\texttt{count} = 0$. 
Fo each $i$, if $A[i]/2$ is in the array, say at location $j$, then $j$
is the first index for which $2A[j]\ge A[i]$, in which case
the program will increase \texttt{count}. This shows that the
program computes the number of indices $i$ for which 
$A[i]/2$ is in the array.

We emphasize that the above argument assumes that all numbers in $A[]$ are
non-negative. In fact, if some numbers are negative, the internal
loop may increase $j$ beyond the range of $A[]$ (although this
is quite easy to correct.)

\medskip
\noindent
(c) The running time analysis is very similar to that for the 
algorithm that computed common elements in two sorted arrays.
Here again we assume that the numbers are non-negative.

The total running time for lines 1,2,3 and 10 is $O(1)$, so it's
negligible. Lines 7,8,9 run in time $O(1)$ and are executed
at most $n$ times, because $i$ is increased at each step. So
the contribution of these lines to the overall running time is
$O(n)$.

The only non-trivial part is the analysis of the running time
of the nested \texttt{while} loops. The key observation is that
in line 6 we always increase $j$ and the value of $j$ never
exceeds $i$, so it never exceeds $n$. Thus the total number of
increases of $j$, over the whole computation, is at most $n$.
This shows that the total contribution of the nested
\texttt{while} loop to the running time is $O(n)$. Putting it
all together, we obtain that the running time is $O(n)$.
\end{solution}

\end{document}
